\documentclass[12pt]{jsarticle}
\usepackage{ascmac,amsmath,amssymb}
\usepackage{amsthm}
\usepackage{cases}
\numberwithin{equation}{section}
%\renewcommand{\baselinestretch}{0.85}

%\pagestyle{empty}

%\oddsidemargin=0cm
%\evensidemargin=0cm
\topmargin -20mm
\textheight 22cm

\newcommand{\diracslash}[1]{#1 \hspace{-5pt}/ \:}
\newcommand{\lambdabar}{\lambda \kern -0.5em\raise 0.6ex \hbox{--}}

\begin{document}
\title{シュレーディンガー方程式の初等的な解}
\author{鴨下正彦}
\date{}
\maketitle
\section{自由粒子}
$V(x)=0$であるシュレーディンガー方程式について考える.
\begin{eqnarray}\label{free particle}
	\nabla^2 u(x) = - \frac{2mE}{\hbar^2} u(x)
\end{eqnarray}
\begin{eqnarray}
	\left|\vec k\right| = k_x^2 + k_y^2 + k_z^2 \equiv \sqrt{\frac{2mE}{\hbar^2}} 
\end{eqnarray}
であるとすると,式(\ref{free particle})の解は,
\begin{eqnarray}\label{solution free}
	u(x) = C e^{i\vec k \cdot \vec r}
\end{eqnarray}
となる.粒子が一辺が$L$の立方体に閉じ込められている(無限に深い井戸型ポテンシャル)場合を考えると解は,
\begin{eqnarray}
	u(x) &&= \frac{1}{L^{\frac{3}{2}}} e^{i\vec k \cdot \vec r} \\
	&& where ~~ k_i = \frac{2\pi}{L} n_i, ~~ i = x,y,z, ~~ n_i \in \mathbb{Z} \nonumber
\end{eqnarray}
またエネルギー固有値は,
\begin{eqnarray}
	E = \frac{\hbar^2}{2m} \left(\frac{2\pi}{L}\right)^2 (n_x^2 + n_y^2 + n_z^2) 
\end{eqnarray}
となる.

\section{調和振動子}
調和振動子のシュレーディンガー方程式
\begin{eqnarray}
	\left[\nabla^2 - \frac{m^2 \omega^2 x^2}{\hbar^2} + \frac{2mE}{\hbar^2}\right]u(x) =  0
\end{eqnarray}
を解く.この解は,$x \to y = \sqrt{\frac{m\omega}{\hbar}}x$と変数変換した後に,$u(y) = h(y) e^{-\frac{y^2}{2}}$とおき,$h(y)$が満たす微分方程式を解くことで得られる.$h(y)$が満たすべき方程式は次式である.
\begin{eqnarray}\label{Hermit}
	\left[\frac{d^2}{d y^2} -2x\frac{d}{d y} + \epsilon-1\right]h_n(y) = 0,\\
	where ~~ \epsilon \equiv \frac{2E}{\hbar\omega}.\nonumber
\end{eqnarray}
式(\ref{Hermit})は,エルミートの微分方程式である.$h_(x)$が無限遠で発散しないためには,
\begin{eqnarray}\label{condition hermit}
	\epsilon-1 &&= 2n \nonumber\\
	&&\therefore E = \left(n+\frac{1}{2}\right)\hbar\omega
\end{eqnarray}
であることが要請される.式(\ref{condition hermit})を満たすとき式(\ref{Hermit})の解はエルミート多項式$H_n(y)$と呼ばれる.
結果として,調和振動子のシュレーディンガー方程式の解は,
\begin{eqnarray}
	u(x) = c_n H_n \left(\sqrt{\frac{m\omega}{\hbar}} x\right) e^{-\frac{m\omega}{\hbar}\frac{x^2}{2}}
\end{eqnarray}


\subsection{エルミート多項式の定義と性質}
エルミート多項式の定義と性質について述べる.
エルミート多項式は様々な定義の仕方があるが,ここでは母関数表示を用いて定義する.
\begin{eqnarray}
	\exp(-z^2+2xz) := \sum_{n=0}^{\infty} \frac{H_n(x)}{n!} z^n.
\end{eqnarray}
これは,
\begin{eqnarray}
	H_n(x) = \frac{n!}{2\pi i}\oint_{C}^{} \frac{\exp(-z^2+2xz)}{z^{n+1}}dz
\end{eqnarray}
とも書くことができる.
母関数表示はロドリーグの公式:
\begin{eqnarray}
	H_n(x) = (-1)^ne^{x^2} \frac{d^n}{dx^n}e^{-x^2}
\end{eqnarray}
と同値である.
\begin{proof}
\begin{eqnarray}
 	\exp(-z^2+2xz) =&& \exp(-(x-z)^2 + x^2) \nonumber\\
	=&& \exp(x^2) \sum_{n=0}^{\infty} \frac{z^n}{n!} \left(\left.\frac{d^n}{dz^n} \exp \left\{-(x-z)^2\right\}\right|_{z=0}\right)\nonumber\\
	=&& \exp(x^2) \sum_{n=0}^{\infty} \frac{z^n}{n!} \left(\sum_{k=0}^{\infty}\frac{1}{k!} \frac{(2k)!}{(2k-n)!}(-1)^n x^{2k-n} \right) \nonumber\\
	=&& \exp(x^2) \sum_{n=0}^{\infty} \frac{z^n}{n!} (-1)^n \frac{d^n}{dx^n} \exp(-x^2) \nonumber\\
	=&& \sum_{n=0}^{\infty} \frac{z^n}{n!} \left[(-1)^n \exp(x^2) \frac{d^n}{dx^n} \exp(-x^2)\right] \nonumber\\
	\therefore&& H_n(x) = (-1)^ne^{x^2} \frac{d^n}{dx^n}e^{-x^2} \nonumber
\end{eqnarray}
\end{proof}
\subsubsection{ロドリーグの公式}
ロドリーグの公式は次のように書くこともできる.
\begin{eqnarray}
	H_{n}(x) = \left(2x-\frac{d}{dx}\right)^n 1
\end{eqnarray}
これは.エルミート関数が満たす漸化式を使うことで示すことができる.

\subsubsection{漸化式}
母関数表示から漸化式を導くことができる.
母関数表示の両辺を$z$で微分することで,
\begin{eqnarray}\label{rec1}
	H_{n+1}(x) = 2xH_n(x) -2nH_{n-1},
\end{eqnarray}
$x$で微分することで,
\begin{eqnarray}\label{rec2}
	H_{n-1}(x) = \frac{1}{2n}\frac{d}{dx}H_n(x)
\end{eqnarray}
が得られる.式(\ref{rec1}),式(\ref{rec2})を連立することで,
\begin{eqnarray}\label{rec3}
	H_{n+1}(x) = \left(2x-\frac{d}{dx}\right)H_n(x)
\end{eqnarray}
が得られる.

\subsubsection{微分方程式の解}
漸化式から,$H_n(x)$に作用する演算子として,
\begin{eqnarray}
	\left(2x-\frac{d}{dx}\right) \frac{1}{2n}\frac{d}{dx} = \frac{1}{2(n+1)}\frac{d}{dx}\left(2x-\frac{d}{dx}\right) = 1
\end{eqnarray}
であることが要請される.
ここから$H_n(x)$は,
\begin{eqnarray}
	\left[\frac{d^n}{dx^n} - 2x \frac{d}{dx} + 2n\right] H_n(x) = 0
\end{eqnarray}
の解であることがわかる.

\subsubsection{直交性}
直交性:
\begin{eqnarray}
	\int_{}^{} H_m(x) H_n(x) e^{-x^2} dx = \sqrt{\pi} n! \delta_{mn}
\end{eqnarray}
が成り立つ.

\section{線形ポテンシャル}
線形ポテンシャル(V字ポテンシャル)のシュレーディンガー方程式
\begin{eqnarray}\label{linear potential}
	\left[\frac{d^2}{dx^2} - \frac{2mk}{\hbar^2}|x| + \frac{2mE}{\hbar^2}\right]u(x) = 0
\end{eqnarray}
を解くことを考える.%ここで,$a \equiv \frac{E}{k}$であるとする.
式(\ref{linear potential})を解く前に見通しを良くすることを考える.

まず,$x\to y = bx$とし,式(\ref{linear potential})を見通しよくするような$b$,$E\to\epsilon$を求める.
\begin{eqnarray}
	&&\left[\left(\frac{dy}{dx}\right)^2\frac{d^2}{dy^2} - \frac{2mk}{\hbar^2}\frac{|y|}{b} + \frac{2mE}{\hbar^2}\right]u(x) = 0 \nonumber\\
	&&\left[\frac{d^2}{dy^2} - \frac{2mk}{\hbar^2}\frac{|y|}{b^3} + \frac{2mE}{\hbar^2 b^2}\right]u(x) = 0
\end{eqnarray}
よって,$b\equiv\left(\frac{mk}{\hbar^2}\right)^{\frac{1}{3}}$,$E\to\epsilon=\frac{m}{\hbar^2b^2} E = \frac{m}{\hbar^2 k^2} E$とすれば式(\ref{linear potential})の見通しが良くなる.
\begin{eqnarray}\label{linear potential 2}
	\left[\frac{d^2}{dy^2} - 2(|y| - \epsilon)\right]u(x) = 0
\end{eqnarray}

式(\ref{linear potential})の解が偶関数または奇関数であることに注目すると,式(\ref{linear potential})の正の領域で方程式を解けばいいことがわかる.
\begin{eqnarray}\label{linear potential 2}
	\left[\frac{d^2}{dy^2} - 2(y - \epsilon)\right]u(x) = 0
\end{eqnarray}

さらに,$y\to z=c(y-\epsilon)$として,式(\ref{linear potential})の見通しを良くすると,
\begin{eqnarray}\label{airy equation}
	&& \left[\left(\frac{dz}{dy}\right)^2 \frac{d^2}{dz^2} - \frac{2}{c} z\right]u(x) = 0\nonumber\\
	&& \therefore \left[\frac{d^2}{dz^2} - \frac{2}{c^3} z\right]u(x) = 0 \nonumber\\
	&& \therefore \left[\frac{d^2}{dz^2} - z\right]u(x) = 0 , ~~ where ~ z=2^{1/3}(y-\epsilon)
\end{eqnarray}

式(\ref{airy equation})の解としてAiry関数が知られている.
\begin{eqnarray}
	u(x) = \mathrm{Ai}(z)
\end{eqnarray}
波動関数が連続かつ滑らかという量子力学的な要請と,関数の偶奇性から,$z=-2^{1/3}\epsilon ~~ (x=y=0)$では$\mathrm{Ai}(z)=0$または$\mathrm{Ai}'(z)=0$であることが要請される.
これを満たす$z$を$z_0$のようにかくとするとエネルギー固有値は,
\begin{eqnarray}
	z_0 = -2^{1/3}\epsilon = - 2^{1/3}\frac{m}{\hbar^2 k^2} E\nonumber\\
	\therefore E = -\frac{z_0}{2^{1/3}}\frac{\hbar^2 k^2}{m}
\end{eqnarray}
と求まる.

$z_0$の値は以下のようになる.
\begin{eqnarray}
	&& \mathrm{Ai}'(z)=0 ~~ \mathrm{for} ~ z=-1.019, -3.249, -4.820, \cdots  ~~ (even)\nonumber\\
	&& \mathrm{Ai}(z)=0 ~~ \mathrm{for}  ~ z=-2.338, 4.088,-5.521, \cdots ~~ (odd)\nonumber
\end{eqnarray}

\subsection{エアリー関数の定義と性質}
エアリー関数の定義と性質を載せる.

\subsection{WKB近似}
WKB近似は,ポテンシャルが位置にゆっくり依存しかつ古典回帰点から離れた領域でよい近似である.井戸型ポテンシャルの解の類推から,波動関数の解は,古典的許容な領域では三角関数,そうではない領域では指数関数減衰する.線形ポテンシャルの解は,これらの領域でWKB近似によって求めた解を古典回帰点近傍で接続する際に用いられる.

\subsubsection{WKB近似}
\begin{eqnarray}\label{shor}
	 \left[ \frac{d^2}{dx^2} + \frac{2m}{\hbar^2}(E-V(x)) \right]u(x) = 0
\end{eqnarray}
を解くことを考える.以下では,$k(x)\equiv\sqrt{\frac{2m}{\hbar^2}(E-V(x))}$とする.解の形を,
\begin{eqnarray}
	u(x) = e^{iW(x)/\hbar}
\end{eqnarray}
と仮定し,式(\ref{shor})に代入すると,
\begin{eqnarray}\label{equation W}
	 e^{iW(x)/\hbar} \left[i\hbar\frac{d^2W(x)}{dx^2} -\left(\frac{dW(x)}{dx}\right)^2 + \hbar^2k^2(x) \right] = 0
\end{eqnarray}
となる.この式で,ポテンシャルがゆっくり変化している$\left(\hbar \left|\frac{d^2W(x)}{dx^2}\right| << \left|\frac{dW(x)}{dx}\right|^2\right)$と仮定すると,
\begin{eqnarray}
	\left(\frac{dW(x)}{dx}\right)^2 =&& \hbar^2 k^2(x) + i\hbar \frac{d^2W(x)}{dx^2} \nonumber\\
	\sim&& \hbar^2 k^2(x) \pm i\hbar^2 \frac{dk(x)}{dx} \nonumber\\
	\therefore \frac{dW(x)}{dx} \sim&&  \pm \sqrt{\hbar^2 k^2(x) \pm i\hbar^2 \frac{dk(x)}{dx}} \nonumber\\
	\sim&&\pm \hbar k(x) \left(1 \pm \frac{i}{2}\frac{\frac{dk(x)}{dx}}{k^2(x)}\right) \nonumber\\
	\therefore W(x) \sim&& \pm \hbar \int_{}^{}k(x)dx + \frac{i}{2}\hbar\ln(k(x))
\end{eqnarray}
のようになる.結果として,
\begin{eqnarray}
	u(x) \sim&& e^{i \left(\pm\int_{}^{}k(x)dx + \frac{i}{2}\hbar\ln(k(x))\right)} \nonumber\\
	=&& \frac{1}{\sqrt{k(x)}} e^{\pm i \int_{}^{}k(x)dx}
\end{eqnarray}
が得られる.

\subsubsection{線形ポテンシャル}
古典的許容な領域と古典的禁制の領域におけるWKB解を接続する.
結果として,古典的禁制領域での波動関数は,
\begin{eqnarray}
	\left\{\frac{2}{[E-V(x)]^{1/4}}\right\}\cos \left[\left(\frac{1}{\hbar}\right) \int_{x_1}^{x} \sqrt{2m \left[E-V(x)\right]} dx - \frac{\pi}{4} \right], ~~ \mathrm {or} \nonumber\\
	\left\{\frac{2}{[E-V(x)]^{1/4}}\right\}\cos \left[\left(-\frac{1}{\hbar}\right) \int_{x}^{x_2} \sqrt{2m \left[E-V(x)\right]} dx + \frac{\pi}{4} \right] \nonumber
\end{eqnarray}
となる.2式が(定数倍で)一致するために,
\begin{eqnarray}
	&&\left(\frac{1}{\hbar}\right) \int_{x_1}^{x} \sqrt{2m \left[E-V(x)\right]} dx - \frac{\pi}{4} = \left(-\frac{1}{\hbar}\right) \int_{x}^{x_2} \sqrt{2m \left[E-V(x)\right]} dx + \frac{\pi}{4} +n\pi \nonumber\\
	&& \therefore \int_{x_1}^{x_2} \sqrt{2m \left[E-V(x)\right]} dx = \left(n+\frac{1}{2}\right) \pi\hbar ~~ (n=0,1,2,\cdots)
\end{eqnarray}
であることが要請される.この要請は,前期量子論のボーア・ゾンマーフェルトの量子化条件と,定数のずれは別として一致する.

\subsubsection{弾んでいるボール}
ポテンシャル
\begin{eqnarray}
	V(x) = 
	\begin{cases}
		mgx ~~ for ~ x>0,\nonumber\\
		\infty ~~ for ~ x<0,\nonumber
	\end{cases}
\end{eqnarray}
のもとで,シュレーディンガー方程式を解く.この問題の解はポテンシャル
\begin{eqnarray}
	V(x) = mg|x|
\end{eqnarray}
における奇関数の解($x=0$で波動関数を0となる)である.
結果として次の量子化条件が得られる.
\begin{eqnarray}
	\int_{0}^{\frac{E}{mg}} \sqrt{2m \left[E-mgx\right]} dx = \left(n-\frac{1}{4}\right)\pi\hbar
\end{eqnarray}
これをエネルギーについて解くと,
\begin{eqnarray}
	\int_{0}^{\frac{E}{mg}} \sqrt{2m \left[E-mgx\right]} dx =&& \sqrt{2mE} \int_{0}^{\frac{E}{mg}} \sqrt{1-\frac{mg}{E}x} dx \nonumber\\
	=&& \sqrt{2mE} \times -\frac{2}{3} \frac{E}{mg} \left[\left(1-\frac{mg}{E}x\right)^\frac{3}{2}\right]_0^\frac{E}{mg} \nonumber\\
	=&& \sqrt{\frac{8}{9mg^2}} E^{\frac{3}{2}} \nonumber\\
	\therefore E_n =&& \frac{\left\{3(n-\frac{1}{4}\pi)\right\}^\frac{2}{3}}{2} (mg^2\hbar^2)^\frac{1}{3}
\end{eqnarray}


\subsubsection{WKB近似が半古典近似だと言われる理由}
$\hbar\to0$の極限で式(\ref{equation W})が,ハミルトンーヤコビ方程式(古典力学の一形式)と同値となる.これが,WKB近似が半古典近似だと言われる所以である.
$W$はハミルトンの特性関数であり,ハミルトンの主関数$S(x,t)$を用いて,
\begin{eqnarray}
	S(x,t) = W(x) - E t \nonumber
\end{eqnarray}
と表される.

\subsubsection{ポテンシャルがゆっくり変化することの定量的な理解}
WKB近似では,ポテンシャルがゆっくり変化するとして近似を行った.そのとき用いた近似式について考察する.
\begin{eqnarray}\label{condition}
	\hbar \left|\frac{d^2W(x)}{dx^2}\right| << \left|\frac{dW(x)}{dx}\right|^2 \Leftrightarrow&& \frac{dk(x)}{dx} << k^2(x)\nonumber\\
	where ~~ k(x)\equiv&&\sqrt{\frac{2m}{\hbar^2}(E-V(x))},\nonumber\\
	\left|\frac{dk(x)}{dx}\right| =&& \frac{1}{2}\sqrt{\frac{2m}{\hbar^2}} \frac{\left|\frac{dV(x)}{dx}\right|}{\sqrt{E-V(x)}},\nonumber\\
	k^2(x) =&&  \frac{2m}{\hbar^2}(E-V(x)).\nonumber\\
	\therefore \hbar \left|\frac{d^2W(x)}{dx^2}\right| << \left|\frac{dW(x)}{dx}\right|^2 \Leftrightarrow&& \lambdabar << \frac{2(E-V(x))}{\left|\frac{dV(x)}{dx}\right|}, \\
	where ~~ \lambdabar \equiv&& \frac{\hbar}{\sqrt{2m(E-V(x))}}. \nonumber
\end{eqnarray}
ここで,$\lambdabar$はドブロイ波長であり,式(\ref{condition})は,ドブロイ波長がポテンシャルの変化を表す長さのスケールよりもはるかに小さいことを意味している.
\end{document}